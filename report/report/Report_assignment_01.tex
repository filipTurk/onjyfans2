%%%%%%%%%%%%%%%%%%%%%%%%%%%%%%%%%%%%%%%%%
% FRI Data Science_report LaTeX Template
% Version 1.0 (28/1/2020)
% 
% Jure Demšar (jure.demsar@fri.uni-lj.si)
%
% Based on MicromouseSymp article template by:
% Mathias Legrand (legrand.mathias@gmail.com) 
% With extensive modifications by:
% Antonio Valente (antonio.luis.valente@gmail.com)
%
% License:
% CC BY-NC-SA 3.0 (http://creativecommons.org/licenses/by-nc-sa/3.0/)
%
%%%%%%%%%%%%%%%%%%%%%%%%%%%%%%%%%%%%%%%%%


%----------------------------------------------------------------------------------------
%	PACKAGES AND OTHER DOCUMENT CONFIGURATIONS
%----------------------------------------------------------------------------------------
\documentclass[fleqn,moreauthors,10pt]{ds_report}
\usepackage[english]{babel}

\graphicspath{{fig/}}




%----------------------------------------------------------------------------------------
%	ARTICLE INFORMATION
%----------------------------------------------------------------------------------------

% Header
\JournalInfo{FRI Natural language processing course 2025}

% Interim or final report
\Archive{Project report} 
%\Archive{Final report} 

% Article title
\PaperTitle{Automatic Generation of Slovenian Traffic News for RTV Slovenija} 

% Authors (student competitors) and their info
\Authors{Filip Turk and Tschimy Aliage Obenga}

% Advisors
\affiliation{\textit{Advisors: Slavko Žitnik}}

% Keywords
\Keywords{Generating traffic reports, Large Language Models, NLP, Prompt Engineering, Fine-tuning, Slovenian traffic news, Automated text generation}

\newcommand{\keywordname}{Keywords}


%----------------------------------------------------------------------------------------
%	ABSTRACT
%----------------------------------------------------------------------------------------

\Abstract{
The abstract goes here.
}

%----------------------------------------------------------------------------------------

\begin{document}

% Makes all text pages the same height
\flushbottom 

% Print the title and abstract box
\maketitle 

% Removes page numbering from the first page
\thispagestyle{empty} 

%----------------------------------------------------------------------------------------
%	ARTICLE CONTENTS
%----------------------------------------------------------------------------------------

\section*{Introduction}
	Traffic reporting is a crucial aspect of public broadcasting, especially for real-time updates on road conditions. Currently, RTV Slovenija relies on students to manually check, filter, and type reports from the Promet.si portal every 30 minutes. This process is time-consuming and prone to inconsistencies.

This project aims to automate traffic news generation using a Large Language Model (LLM). The approach includes leveraging prompt engineering techniques, defining evaluation criteria, and fine-tuning an LLM to improve accuracy and relevance. The generated reports must align with RTV Slovenija’s guidelines, ensuring clarity, conciseness, and correctness in road naming and event significance.


%------------------------------------------------

\section*{Existing Solutions and Related Work}

\subsection*{Traffic News Automations}
Automated traffic reporting systems have been developed worldwide, primarily using Natural Language Processing (NLP) and Machine Learning (ML) techniques. Some existing solutions include:

\begin{enumerate}[noitemsep] 
	\item Google Maps Traffic Alerts: Uses real-time data and machine learning to generate concise traffic updates
    
	\item Waze Traffic Reports: Crowdsourced traffic conditions, automatically summarized for users.
    
	\item AI-driven News Generation: Organizations like OpenAI and Google have explored AI-generated news reports in sports, finance, and weather domains.
\end{enumerate}

These solutions utilize structured traffic data and LLMs for automated summarization. However, fine-tuning on local datasets, such as Slovenian road networks and traffic terminology, remains a challenge.

\subsection*{Large Language Models in Text Generation}

Recent advancements in LLMs like GPT-4, LLaMA, and T5 have demonstrated strong capabilities in text summarization and structured data interpretation. Studies have shown that fine-tuning domain-specific models improves text coherence and factual accuracy.

\section*{Initial Corpus Analysis}

The dataset for this project consists of structured traffic data from Promet.si, formatted in Excel. A preliminary analysis reveals the following key attributes:

Road Names and Locations: Entries specify highways, regional roads, and key urban intersections.

Incident Types: Traffic congestion, road closures, weather-related disruptions, and accidents.

Time Stamps: Time of report updates, crucial for real-time relevance.

Severity and Impact: Differentiates minor delays from major traffic disruptions.

Initial observations suggest that filtering out redundant or low-impact reports will be necessary to maintain concise news summaries. To achieve this, categorizing incidents based on importance will help prioritize critical updates. Categories could include:

\begin{enumerate}[noitemsep] 
	\item High Priority: Major accidents, full road closures, severe weather disruptions
    
	\item Medium Priority: Traffic congestion, partial road closures, significant delays
    
	\item Low Priority: Minor roadworks, temporary disruptions with minimal impact
\end{enumerate}

This structured approach will enhance the efficiency of traffic news generation by ensuring that significant events receive appropriate attention.

\section*{Initial Ideas and Approach}

\subsection*{Prompt Engineering}

The first step involves experimenting with LLMs through structured prompts to generate effective traffic summaries. Potential techniques include:

Using role-based prompting (e.g., “You are a Slovenian traffic news reporter”)

Applying structured templates to ensure consistent output

Fine-tuning prompt phrasing to optimize summarization quality

\subsection*{Evaluation Criteria Definition}

To ensure quality traffic reports, evaluation metrics will focus on:

\begin{enumerate}[noitemsep] 
	\item Accuracy: Correct road names and locations
    
	\item Conciseness: Keeping reports within an optimal word count

        \item Format: Ensuring the report fit the provided format
\end{enumerate}


\subsection*{Parameter-Efficient Fine-Tuning}

Once prompt engineering is refined, fine-tuning an LLM (such as GPT-3.5 or T5) with a domain-specific dataset will be explored. LoRA (Low-Rank Adaptation) or Adapter-based methods will be considered to optimize training efficiency.

\subsection*{Interactive Testing Interface}

A web-based or command-line interface will be developed to allow interactive testing of generated reports before full automation.


%----------------------------------------------------------------------------------------
%	REFERENCE LIST
%----------------------------------------------------------------------------------------
\bibliographystyle{unsrt}
\bibliography{report}


\end{document}